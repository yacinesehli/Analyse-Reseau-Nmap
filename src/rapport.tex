\documentclass{article}
\usepackage[utf8]{inputenc}
\usepackage[T1]{fontenc}
\usepackage[french]{babel}
\usepackage{graphicx}
\usepackage[margin=2.5cm]{geometry}
\usepackage{hyperref}
\usepackage{listings}
\usepackage{xcolor}
\usepackage{fancyhdr}
\usepackage{titlesec}
\usepackage{float}

% --- Couleurs et Styles ---
\definecolor{KaliBlue}{HTML}{2B5876}
\definecolor{codegray}{rgb}{0.5,0.5,0.5}

\lstdefinestyle{terminal}{
  basicstyle=\small\ttfamily,
  backgroundcolor=\color{white},
  frame=single,
  rulecolor=\color{KaliBlue},
  breaklines=true,
  keywordstyle=\color{blue},
  captionpos=b,
  numbers=left,
  numberstyle=\tiny\color{codegray}
}
\lstset{style=terminal}

\titleformat{\section}[block]{\color{KaliBlue}\large\bfseries\filcenter}{\thesection}{1em}{}

% --- En-tête et Pied de page ---
\pagestyle{fancy}
\fancyhead[L]{\textbf{Analyse Réseau Nmap/Zenmap}}
\fancyhead[R]{\textbf{Yacine SEHLI}}
\fancyfoot[C]{\thepage}

% --- TITRE ---
\title{\textbf{\Huge Analyse de Réseau Local avec Nmap et Zenmap}}
\author{Yacine SEHLI \\ \small Étudiant passionné en cybersécurité et réseaux}
\date{Octobre 2025}

\begin{document}

\maketitle

\begin{abstract}
Ce projet présente une analyse complète du réseau local réalisée à l'aide des outils \textbf{Nmap} et \textbf{Zenmap}. L'objectif était de découvrir les hôtes actifs, d'identifier les ports ouverts et de reconnaître les systèmes d'exploitation afin d'évaluer la sécurité du réseau. Le projet a été mené sur une période de deux jours, en utilisant un environnement Kali Linux configuré sur une machine virtuelle.
\end{abstract}

\newpage
\tableofcontents
\newpage

\section{Introduction}
[cite_start]La sécurité des réseaux est un enjeu majeur dans le domaine de la cybersécurité[cite: 1, 3, 17]. [cite_start]Pour tout administrateur ou analyste SOC, la connaissance des outils de scan réseau est essentielle pour détecter les vulnérabilités potentielles[cite: 18]. [cite_start]\textbf{Nmap} (Network Mapper) est un des outils les plus puissants pour cette tâche, et \textbf{Zenmap} constitue son interface graphique conviviale[cite: 2, 19].

[cite_start]Ce rapport documente les différentes étapes de la mise en œuvre d'un scan réseau local, effectué sur l'adresse \texttt{192.168.1.1}, ainsi que l'analyse et l'interprétation des résultats obtenus[cite: 20].

\section{Présentation de Nmap et Zenmap}

\subsection{Nmap}
[cite_start]Nmap est un outil open source utilisé pour l'exploration réseau et l'audit de sécurité[cite: 24]. [cite_start]Il permet de découvrir les hôtes présents sur un réseau, les ports ouverts, les services disponibles et les systèmes d'exploitation associés[cite: 25].

\subsection{Zenmap}
[cite_start]Zenmap est l'interface graphique officielle de Nmap[cite: 27]. Elle facilite la création, la sauvegarde et la visualisation des scans, notamment à travers des profils prédéfinis comme le << Intense Scan >>.

\section{Objectifs du projet}
[cite_start]Les objectifs principaux de ce projet étaient[cite: 30, 31, 32, 33, 34]:
\begin{itemize}
    \item Comprendre le fonctionnement des commandes Nmap et Zenmap.
    \item Effectuer un scan complet d'un hôte local (\texttt{192.168.1.1}).
    \item Identifier les ports et services ouverts.
    \item Analyser le type de système détecté.
    \item Présenter les résultats dans un rapport professionnel.
\end{itemize}

\section{Environnement et configuration}

\subsection{Matériel et logiciel utilisés}
\begin{itemize}
    [cite_start]\item \textbf{Système d'exploitation :} Kali Linux 2025 [cite: 38]
    [cite_start]\item \textbf{Outils :} Nmap 7.95 et Zenmap GUI [cite: 38]
    [cite_start]\item \textbf{Virtualisation :} Oracle VirtualBox [cite: 38]
    [cite_start]\item \textbf{Cible :} 192.168.1.1 (Routeur / Passerelle locale) [cite: 39]
\end{itemize}

\subsection{Préparation du scan}
[cite_start]Le scan a été réalisé dans un environnement isolé, sans impact sur le réseau de production[cite: 41]. [cite_start]Les permissions administrateur (\texttt{sudo}) ont été utilisées pour garantir un scan complet et précis (SYN Scan)[cite: 42].

\section{Méthodologie et commandes utilisées}

\subsection{Scan Nmap en ligne de commande}
[cite_start]Nous avons d'abord effectué un scan complet de tous les ports TCP pour identifier la surface d'attaque globale[cite: 46, 47, 48].

\begin{lstlisting}[caption={Scan complet des ports TCP}]
nmap -p0-65535 192.168.1.1
\end{lstlisting}

\begin{figure}[H]
    \centering
    \includegraphics[width=0.9\textwidth]{../evidence/nmap_full_scan.png}
    \caption{Résultat du scan Nmap en ligne de commande (Ports ouverts).}
\end{figure}

\subsection{Scan approfondi avec détection d'OS et de services}
[cite_start]Ensuite, un scan plus agressif a été lancé pour identifier les versions et l'OS[cite: 49, 50, 51].

\begin{lstlisting}[caption={Scan avancé avec détection complète}]
nmap -p1-65535 -T4 -A -v 192.168.1.1
\end{lstlisting}

[cite_start]\textbf{Les options utilisées[cite: 52, 53, 54, 55, 56]:}
\begin{itemize}
    \item \texttt{-p1-65535} : scan de tous les ports TCP.
    \item \texttt{-T4} : vitesse du scan augmentée (Aggressive).
    \item \texttt{-A} : détection de l'OS, des versions de services et traceroute.
    \item \texttt{-v} : mode verbeux (affiche les détails en temps réel).
\end{itemize}

\section{Résultats et interprétation}

\subsection{Analyse des Ports et Services}
[cite_start]Les ports suivants ont été détectés ouverts sur l'hôte[cite: 60, 61, 62, 63, 67, 68, 69]:

\begin{table}[H]
\centering
\begin{tabular}{|l|l|l|}
\hline
\textbf{Port} & \textbf{Service} & \textbf{Description} \\ \hline
53/tcp & Domain & Service DNS (dnsmasq) \\ \hline
80/tcp & HTTP & Serveur Web (Boa httpd) \\ \hline
139/tcp & NetBIOS-ssn & Samba (Partage de fichiers) \\ \hline
445/tcp & Microsoft-DS & Samba (Partage de fichiers) \\ \hline
2000/tcp & Cisco-SCCP & Protocole VoIP (Skinny) \\ \hline
5060/tcp & SIP & Session Initiation Protocol (VoIP) \\ \hline
\end{tabular}
\caption{Tableau récapitulatif des services découverts}
\end{table}

\begin{figure}[H]
    \centering
    \includegraphics[width=0.9\textwidth]{../evidence/nmap_service_scan.png}
    \caption{Détail des versions de services détectées (Samba, Boa httpd).}
\end{figure}

\subsection{Analyse Zenmap}
[cite_start]L'interface Zenmap a permis de visualiser la progression et les détails de l'OS[cite: 144].

\begin{figure}[H]
    \centering
    \includegraphics[width=0.9\textwidth]{../evidence/zenmap_results.jpg}
    \caption{Résultat Zenmap : Détection de l'OS (Oracle VirtualBox / Linux).}
\end{figure}

[cite_start]\textbf{Interprétation :} Les services détectés (SIP, SCCP, DNS) indiquent que la machine cible agit probablement comme un équipement de communication (Passerelle VoIP) ou un routeur virtuel (VirtualBox NAT)[cite: 71, 72].

\section{Analyse critique et bonnes pratiques}
[cite_start]Lors de cet audit, les bonnes pratiques suivantes ont été respectées[cite: 197, 198, 199]:
\begin{itemize}
    \item Vérifier les permissions et la portée du scan avant exécution.
    \item Utiliser les options Nmap de façon responsable.
    \item Documenter chaque commande pour assurer la traçabilité.
    \item Compléter l’analyse avec d’autres outils (Wireshark, Nikto) pour une vision plus approfondie.
\end{itemize}

\section{Conclusion}
[cite_start]Ce projet m’a permis de maîtriser les fonctionnalités essentielles de Nmap et Zenmap en seulement deux jours[cite: 200]. [cite_start]L’analyse m’a offert une meilleure compréhension de la topologie réseau, de l’identification des ports et des services, ainsi que de la détection d’OS[cite: 201].

[cite_start]À l’avenir, je compte approfondir mes connaissances en automatisant les scans via des scripts NSE et en couplant les résultats à des outils d’analyse de vulnérabilités[cite: 202].

\end{document}